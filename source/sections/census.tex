% !TEX root = ../main.tex

\section{A Social Media Census}
\label{sec:census}

In her 2018 article for the Harvard Law Review~\cite{Klonick2018}, Kate Klonick
traces the early history of content moderation at Facebook, Twitter, and
YouTube. She also discusses their privileged position when it comes to speech,
and proposes a new label, and also legal role, to go along with that,
\emph{governance}. Her claims about social media's grounding in the tradition of
free speech seem wilfully oblivious to capitalist reality, which, for example,
grants Mr Zuckerberg over 50\% of voting shares for Meta and hence autocratic
control~\cite{LauricellaNorton2021}. But her notion of governance is compelling.
Alas, governance requires accountability requires transparency. This section
tries to provide just that by comparing non-Chinese social media based on their
content policies and transparency reports. For good measure, I am also including
the \EU's Digital Services Act (\DSA)~\cite{EuropeanParliamentAndCouncil2022}.

To determine which social media to include in the census, I started with
Buffer's list of the 20 most popular platforms~\cite{Lua2022}. I treat Facebook
and Instagram as one platform, since Meta sanctions users based on their posts
to either social network. I drop the five platforms targeting China, since they
are unlikely to adhere to Klonick's governance model. I also drop Facebook
Messenger, Microsoft Teams, Skype, and WhatsApp, since they mostly enable
private communication between users. While that arguably is the predominant use
case for Telegram, too, it stickers, channels, and bots are public and require
moderation, as even Telegram acknowledges in their \FAQ. Hence, Telegram
remains. Finally, I add Tumblr because that social network has become a popular
destination for users leaving Twitter as well as employees laid off by Twitter
in the wake of Elon Musk's takeover of the firm.

To determine the criteria for the census, I iterated a few times over previous
studies~\cite{BradfordGriselea2019,CrockerGebhartea2019}, best practices
recommendations~\cite{AccessNowACLUFoundationOfNorthernCaliforniaea2021}, and
the \EU's \DSA. I eventually settled on the following hierarchy of five aspects
with 1--4 actual criteria per aspect and 14 criteria total. I found that the
arrangement makes it easier to reason about the completeness of criteria; I hope
that holds for you, too. The emoji serve as column headers in the results table.

% \hfill pushes the nested enumerations onto the next line!

\begin{description}
\item[The violative content itself]\hfill
    \begin{enumerate}
        \item \emo{tabs}~Prevalence of violative content measured as fraction
            of all content
        \item \emo{rulertriangle}~Reach of violative content measured in time or
            better views
    \end{enumerate}
\item[The entity labelling violative content as such]\hfill
    \begin{enumerate}[resume]
        \item \emo{temple}~Government inquiries and removal requests; also
            intellectual property claims
        \item \emo{enragedface}~Platform users
        \item \emo{robot}~Algorithmic enforcement
        \item \emo{judge}~Human moderation, including its mental health impact
    \end{enumerate}
\item[Sanctions on content and users]\hfill
    \begin{enumerate}[resume]
        \item \emo{label}~Warning labels (increasingly used for misinformation)
        \item \emo{fog}~Visibility reduction and blacklists (aka shadow banning)
        \item \emo{wastebasket}~Content removal
        \item \emo{biohazard}~Sanctions on users
    \end{enumerate}
\item[Due process protections]\hfill
    \begin{enumerate}[resume]
        \item \emo{pager}~Meaningful notification
        \item \emo{scales}~Appeals and reversals
        \item \emo{eye}~External oversight, including conflict resolution and auditing
    \end{enumerate}
\item[Cultural proficiency and fluency]\hfill
    \begin{enumerate}[resume]
        \item \emo{globeafrica}~Different countires and cultures
    \end{enumerate}
\end{description}

Table~\ref{table:governance} shows the results of the social media census. Eight
platforms have published content policies and release transparency reports
covering their content moderation. However, three platforms fall short of even
these basic niceties of governance. Telegram has no policy and makes no
transparency disclosures. Quora has a policy but makes no disclosures. Finally,
Tumblr has a policy but discloses only governmental requests and intellectual
property claims. While the platforms' prohibitions have strikingly similar
names, prior work has demonstrated substantial differences in definitions and
enforcement~\cite{FieslerJiangea2018,PaterKimea2016}.

\begin{table}
\caption{A survey of governance practices for social media. \emph{Account
    Sanctions} are coded W for warning, a number for as many days of forced
    timeout, and X for account suspension. The \emo{robot} robot and \emo{judge}
    judge columns share the \emo{keycapone} number one when transparency reports
    do not distinguish between in-house automated and human review.}
\label{table:governance}
\begin{tabular}{LCcCc@{\;}c@{\quad}c@{\;}c@{\;}c@{\;}c@{\quad}c@{\;}c@{\;}c@{\;}c@{\quad}c@{\;}c@{\;}c@{\quad}cC}

\multicolumn{2}{C}{\textbf{Platform}} & \textbf{Account} & \textbf{Latest}
& \multicolumn{14}{c}{\textbf{Included in Transparency Report\T}} & \\

\multicolumn{2}{C}{\textbf{Name \& Launch}}
& \textbf{Sanctions}
& \textbf{Report}
& \emo{tabs}
& \emo{rulertriangle}
& \emo{temple}
& \emo{enragedface}
& \emo{robot}
& \emo{judge}
& \emo{label}
& \emo{fog}
& \emo{wastebasket}
& \emo{biohazard}
& \emo{pager}
& \emo{scales}
& \emo{eye}
& \emo{globeafrica} \T\B
& \multirow{-2}{*}{\emo{floppydisk}} \\ \hline

\href{https://transparency.fb.com/policies/community-standards/}{FB/IG}
& 2003
& \href{https://transparency.fb.com/enforcement/taking-action/restricting-accounts/}{W,1,3,7,30,X}
& \href{https://transparency.fb.com/data/community-standards-enforcement/}{Q3 2022}
& \MK & & \MK & \MK & \MKONE &
& & \MK & & \MK & \MK & \MK & &
\href{https://transparency.fb.com/sr/community-standards/}{CSV} \T\\

\href{https://www.linkedin.com/legal/professional-community-policies}{LinkedIn}
& 2003
& & \href{https://about.linkedin.com/transparency/community-report}{H1 2022}
& & & \MK & \MK & \MKONE
& & & \MK & & & & & & \\

\href{https://policy.pinterest.com/en/community-guidelines}{Pinterest}
& 2010
& & \href{https://policy.pinterest.com/en/transparency-report}{H1 2022}
& & \MK & \MK & \MK & \MK & \MK & &
& \MK & \MK & & \MK & & & \\

\href{https://help.quora.com/hc/en-us/articles/360000470706-Platform-Policies}{Quora}
& 2010
& & & & & & & & & & & & & & & & & \\

\href{https://www.redditinc.com/policies/content-policy}{Reddit}
& 2005
& \href{https://www.redditinc.com/policies/transparency-report-2021-2/}{W,3,7,X}
& \href{https://www.redditinc.com/policies/transparency-report-2021-2/}{2021}
\hidden{\,/\,\href{https://www.redditinc.com/policies/mid-year-transparency-report-2022}{H1
2022}
\href{https://www.reddit.com/r/redditsecurity/comments/rgikn1/q3_safety_security_report/}{Q3
2021}}
& \MK & & \MK & \MK & \MK & \MK & & & \MK & \MK & \MK & \MK & & \MK & \\

\href{https://values.snap.com/privacy/transparency/community-guidelines}{Snap}
& 2011
& & \href{https://values.snap.com/privacy/transparency}{H1 2022}
& & \MK & \MK & \MK & & & &
& \MK & \MK & & & & \MK &
\href{https://assets.ctfassets.net/kw9k15zxztrs/4aBF69gFiv4l9x7LzAgY8X/578e44f80fc51d0ebc52e04fe3d8c857/Snap_H1_2022_TR.csv}{CSV}\\

Telegram & 2013 & & & & & & & & & & & & & & & & & \\

\href{https://www.tiktok.com/community-guidelines}{TikTok}
& 2017
& & \href{https://www.tiktok.com/transparency/en/community-guidelines-enforcement-2022-3/}{Q3 2022}
& \MK & \MK & \MK & \MK & \MKONE & &
& \MK & \MK & & & & \MK &
\href{https://sf16-va.tiktokcdn.com/obj/eden-va2/nuvlojeh7ryht/Transparency_CGE_2022Q3/English_CGE_2022Q3.xlsx}{XLS} \\

\href{https://www.tumblr.com/policy/en/community}{Tumblr}
& 2007
& & \href{https://transparency.automattic.com/tumblr/}{H1 2022}
& & & \MK & & & & & & & & & & & & \\

\href{https://help.twitter.com/en/rules-and-policies/twitter-rules}{Twitter}
& 2006
& \href{https://help.twitter.com/en/rules-and-policies/enforcement-options}{W/0.5--7/X}
& \href{https://transparency.twitter.com/en/reports/rules-enforcement.html#2021-jul-dec}{H2 2021}
& & \MK & \MK & \MK & \MKONE &
& & \MK & \MK & \MK & & & &
\href{https://transparency.twitter.com/content/dam/transparency-twitter/download/rules_enforcement_test.zip}{CSV} \\

\href{https://www.youtube.com/howyoutubeworks/policies/community-guidelines/}{YouTube}
& 2005
& \href{https://support.google.com/youtube/answer/2802032}{W,7,14,X}
& \href{https://transparencyreport.google.com/youtube-policy/removals}{Q3 2022} \B
& \MK & \MK & \MK & \MK & \MK & \MK & & & \MK & \MK & & \MK & & \MK & \\
\hline

\multicolumn{4}{L}{\textbf{Required by EU Digital Services Act\T}} & &
& \euflag & \euflag & \euflag & & \euflag & \euflag & \euflag & \euflag &
\euflag & \euflag & \euflag & & \euflag \\

\end{tabular}
\end{table}

Still, a few prohibitions do stick out. Pinterest's ``Harassment and criticism''
is chilling in its pathologizing overreach. Then again, Tumblr's
``Misattribution or Non-Attribution'' is endearing --- but probably not that
practical. Frankly, I am troubled by the many prohibitions against
misinformation~\cite{Douek2021,Masnick2019}. While such prohibitions may seem
like a good idea in theory, the practice of outsourced content reviewers
deciding in 30 seconds which statements are science fact and which are science
disinformation should disabuse anyone of the idea. In addition to content
moderators, the very authorities deciding on what counts as disinformation are
often slow to adjust to new information and/or have their own fact-defying
agendas. Remember that, during the first few months of the pandemic, both WHO
and CDC falsely claimed that facemasks were ineffective against Covid-19.
Finally, censoring bona fide disinformation may reduce people's exposure but it
also does nothing towards changing their minds, with the likely result that
people start mistrusting the platform.

When it comes to sanctioning accounts, \emph{all} social media (besides
Telegram) fall in one of two groups. The first group seems to model their
sanctions on the Queen of Hearts from \emph{Alice's Adventures in Wonderland},
enthusiastically screaming ``Off with their heads!'' and then even figuratively
following through~\cite{Carroll2008}. The second group manages to improve on
that (low) standard, but only barely, by modelling their sanctions on
California's 1994 ``three strikes'' law~\cite{Vitiello2002}. A decade later, the
law accounted for a quarter of the state's prisoners, but did not curb
crime~\cite{BrownJolivette2005}, increased homicides~\cite{MarvellMoody2001},
and led to even more Black men being locked up for even
longer~\cite{BrownJolivette2005}. Three decades after enactment, the most
salient change is that the law accounts for a third of all the state's
prisoners~\cite{BirdGillea2022}. In short, the uniformity and severity of these
sanctions put all social media squarely into the stochastic penal colony.

That conclusion does not change if I avoid the damning analogues and restrict
the framing to content moderation only: When it comes to judging the severity of
violative content (with possible exception of \CSAM\ and terrorism), Americans,
including moderating practitioners, and international audiences alike reason
along many fine-grained axes and thereby make subtle
distinctions~\cite{JiangScheuermanea2021,ScheuermanJiangea2021}. International
audiences only add more variability, depending on culture. And to all that
subtle reasoning and weighing, social media respond with one emphatic ``Off with
their heads!'' Yeah, that's appropriate. Furthermore, \AI{}-based enforcers
increasingly intercept content \emph{before} it is posted. That makes it
significantly harder to justify \emph{any} sanction. Yet platforms just ignore
that inconvenient fact.

In addition to confirming my hypothesis about being overly punitive, the
sparsity of checkmarks, especially when it comes to disclosing \emph{all}
sanctions, notifying users, and due process, provides a rude counterpoint to
Klonick's cheerful enthusiasm for their governance. Worse, as the \emph{Launch}
years make clear, all but TikTok had a decade or longer for making the necessary
investments.

That won't do for much longer, unless they are willing to forgo the European
market. As the bottom row illustrates, the \EU's \DSA\ has extensive
requirements for transparency reporting --- and for yearly independent audits,
too. The timetable for becoming compliant is tight. By now, all platforms must
have reported their user numbers to the \EU, so that the Commission can
determine which of the \DSA's tiered requirements apply. Once the Commission has
designated a platform as very large (the \DSA's upmost tier), the firm has only
four months to comply with the law. All other platforms have until February
2024. Penalties for non-compliance can be steep: Each fine is capped at 6\% of
global revenue for the previous year.

Missing from the census table is that most social media platforms' transparency
disclosures lack organization and get in the way of additional analysis. It's
utterly ridiculous that five out of nine technology firms can't be bothered to
release their data in machine-readable form. Thankfully, the \DSA\ should put an
end to that, too. But the lack of organization makes it far too easy to get lost
in platforms' reports. That's just why I iterated over the census criteria
several times, updating Table~\ref{table:governance} for every iteration. In my
mind, Pinterest's and YouTube's disclosures stand out, with YouTube's resembling
my own. For historical reasons, sections on government requests and intellectual
property claims still are separate, in their own reports even. That only gets in
the way of understanding. It's time to produce one single governance report that
pays more attention to organization as well.

Finally, social media's transparency reports tell vastly different stories about
the role of \AI. As already discussed in Section~\ref{sec:tweet:da:fe}, Twitter
largely is in public denial about its aggressive use of algorithms. In contrast,
Pinterest is very exacting and distinguishes between automated and hybrid
detection. In the latter case, algorithms pick violative content too, but only
humans get to make a decision. Meanwhile, TikTok seems to be playing catch up
with Meta and YouTube in automating almost all content moderation. From Q3~2020
to Q3~2021, the share of automatically removed video clips grew from 7.8\% to
33.9\% of all removed clips. After plateauing at that level for another two
quarters, it resumed it rapid growth and stood at 48.0\% for Q3~2022. Will they
make it to 99\% over the next two years?


\subsection{Validating the Transparency Data}
\label{sec:census:validation}

% !TEX root = ../main.tex

\newpage
\section{Homo Infanti Lupus}
\label{adx:csam}

Since transparency disclosures, by definition, are based on platform-internal
data, the public must trust that platforms correctly collect and report the
data. At least for the \EU, that will change as the \DSA's yearly audit
requirement takes effect in February 2024. Still, I was curious if there were
other opportunities for independently validating such disclosures.

As it turns out, there is: In the United States, child sexual abuse material
(\CSAM) must be reported to a clearinghouse, the National Center for Missing and
Exploited Children (\NCMEC). \NCMEC{} makes its own, yearly transparency
disclosures. They include a breakdown of internet platforms' report counts.
Hence, it should be possible to cross-check \NCMEC's data with that released by
social media in their own transparency disclosures.

For practical reasons, I limited the comparison of \CSAM\ disclosures to Meta
and Google. I include Meta because, at over 90\% of all reports made to \NCMEC,
it appears to be a ``hotbed of \CSAM''~\cite{Hitt2021} (and not Pornhub, as
falsely asserted~\cite{Brown2020,Grant2020} with devastating
consequences~\cite{Celarier2021,Dickson2020,Harris2021,Stoya2021,TheIndependent2021b}
by a certain New York Times columnist, who made common cause with anti-porn
crusaders~\cite{Hitt2020a} associated with a white supremacist
church~\cite{Halley2021,ProducerX2020} after pulling a similar stunt
before~\cite{Bass2014,Brown2019,Dickson2014,HuffPost2012,Martin2012,Masnick2017,Talusan2017}).
While Google is a distant second, I originally included the firm because it uses
the same two \CSAM\ detection systems as Meta~\cite{Allen2011,Davis2018}:
Microsoft's PhotoDNA for detecting previously known instances of \CSAM\ based on
perceptual hashing and Google's own Content Safety \API{} for detecting
previously unknown instances of \CSAM\ based on a machine learning model.

Table~\ref{table:csam} summarizes \emph{all} \CSAM\ disclosures by the three
entities~\cite{NcmecByPlatform2019,NcmecByPlatform2020,NcmecByPlatform2021}. In
the language of \NCMEC, which I adapt for the rest of this section, a
\emph{report} includes one or more \emph{pieces} of \CSAM, i.e., photos or
videos. \NCMEC\ makes yearly disclosures of report counts, Google makes
semiannual disclosures of both report and piece counts, and Meta makes quarterly
disclosures of only piece counts. That already implies that only \NCMEC's and
Google's data are comparable. As the table shows, they are reasonably close,
with a maximum yearly difference of -0.6\% between \NCMEC's report count for
Google and Google's own report count.

\begin{table}
\caption{\CSAM{} disclosures by \NCMEC, Google, and Meta from 2018 to 2022.}
\label{table:csam}
\begin{tabular}{cRrrRRrRrRr}

& \multicolumn{5}{>{\columncolor[gray]{.91}}c}{\textbf{NCMEC}\T}
&
& \multicolumn{3}{>{\columncolor[gray]{.91}}c}{\textbf{Google}}
& \multicolumn{1}{c}{\textbf{Meta}} \\

\multirow{-2}{*}{\textbf{Year}}
& \multicolumn{1}{>{\columncolor[gray]{.91}}c}{\textbf{Reports}}
& \multicolumn{2}{c}{\textbf{From Meta}}
& \multicolumn{2}{>{\columncolor[gray]{.91}}c}{\textbf{From Google}\B}
& \multicolumn{1}{c}{\multirow{-2}{*}{$\bm{\Delta}$}}
& \multicolumn{1}{>{\columncolor[gray]{.91}}c}{\textbf{Reports}}
& \multicolumn{1}{c}{$\bm{\Pi$}}
& \multicolumn{1}{>{\columncolor[gray]{.91}}c}{\textbf{Pieces}}
& \multicolumn{1}{c}{\textbf{Pieces}} \\
\hline
& & & & & & & & & & \T\\
& & & & & & & & & & \\
& & & & & & & & & & 9,000,000 \\
\multirow{-4}{*}{2018} & & & & & & & & & & \B 7,200,000 \\
\hline
& & & & & & & & & & \T 5,800,000 \\ \cline{11-11}
& & & & & & & & & & 7,426,200 \\
& & & & & & & & & & 12,155,800 \\
\multirow{-4}{*}{2019}
& \multirow{-4}{*}{16,836,694}
& \multirow{-4}{*}{94.3\%} & \multirow{-4}{*}{15,884,511}
& \multirow{-4}{*}{2.7\%}  & \multirow{-4}{*}{449,283}
& & & & & \B 13,986,400 \\
\hline
& & & & & & & & & & \T 9,500,000 \\
& & & & & &
& \multirow{-2}{*}{182,556} & \multirow{-2}{*}{8.4×} & \multirow{-2}{*}{1,533,536}
& 2,958,200 \\
& & & & & & & & & & 10,770,600 \\
\multirow{-4}{*}{2020}
& \multirow{-4}{*}{21,447,786}
& \multirow{-4}{*}{94.7\%} & \multirow{-4}{*}{20,307,216}
& \multirow{-4}{*}{2.5\%}  & \multirow{-4}{*}{546,704}
& \multirow{-4}{*}{+0.2\%}
& \multirow{-2}{*}{365,319} & \multirow{-2}{*}{8.0×} & \multirow{-2}{*}{2,904,317}
& \B 4,958,900 \\
\hline
& & & & & & & & & & \T 5,812,400 \\ \cline{11-11}
& & & & & &
& \multirow{-2}{*}{412,141} & \multirow{-2}{*}{8.3×} & \multirow{-2}{*}{3,413,673}
& 27,000,000 \\
& & & & & & & & & & 22,800,000 \\
\multirow{-4}{*}{2021}
& \multirow{-4}{*}{29,157,083}
& \multirow{-4}{*}{92.2\%} & \multirow{-4}{*}{26,885,302}
& \multirow{-4}{*}{3.0\%}  & \multirow{-4}{*}{875,783}
& \multirow{-4}{*}{-0.6\%}
& \multirow{-2}{*}{458,178} & \multirow{-2}{*}{7.2×} & \multirow{-2}{*}{3,282,824}
& \B 22,400,000 \\
\hline
& & & & & & & & & & \T 18,000,000 \\
& & & & & &
& \multirow{-2}{*}{1,044,277} & \multirow{-2}{*}{6.4×} & \multirow{-2}{*}{6,698,201}
& 21,600,000 \\
& & & & & & & & & & 31,400,000 \\
\multirow{-4}{*}{2022}
& & & & & & & & & & \\
\end{tabular}
\end{table}

Unfortunately, Meta's statistics aren't just incomparable to \NCMEC's. They also
suffer from several other short-comings. Meta has disclosed but not explained
most of them. I discovered a major data quality issue surepitiously after
re-implementing my original spreadsheet-based analysis with Pandas in Python and
while comparing results:

\begin{enumerate}
\item As indicated by the horizontal lines between Q1/Q2 2019 and Q1/Q2 2021,
    Meta changed metrics twice. At first, the firm disclosed ``Child Nudity \&
    Sexual Exploitation,'' which covers child nudity, child physical abuse, and
    child sexual abuse, for Facebook only. In Q2 2019, it started reporting the
    same metric for Instagram, too. In Q2 2021, it switched to reporting \CSAM{}
    by itself, though the 4.6$\times$ increase despite the new metric covering
    less pieces is surprising.
\item No other social media platform is rounding transparency numbers, yet Meta
    is obviously doing so. Worse, it does so with different precision for
    Facebook and Instagram.
\item For 2019 and 2020, \NCMEC{} disclosed a single report count for Meta. For
    2021, it disclosed separate counts for Facebook, Instagram, and WhatsApp.
    \emph{WhatsApp}? Meta has yet to make a transparency disclosure for
    WhatsApp.
\item I prepared a first version of the above table based on Meta's Q2 2022
    disclosure in the late summer of 2022. During the following winter, I
    replaced the original spreadsheet with Python code that used Meta's Q3 2022
    disclosure. When I compared results, the numbers for Q4 2020 differed. After
    checking my work and finding no issues, I compared the data in Meta's CSV
    files and discovered that 113 data points across several categories
    including \CSAM{} had changed, 77 for Q4 2020, 3 for Q1 2021, 4 for Q2 2021,
    and the rest for Q2 2022. Furthermore, out of the 77 updated metrics for Q4
    2020, the 58 updated absolute counts changed substantially, from -50.0\% to
    -0.1\%. Yet neither the transparency report nor the blog post announcing the
    report's release mention these changes. That seems sloppy at the very best.
\item In February 2021, Antigone Davis, Meta's Global Head of Safety, reported
    in a blog post~\cite{Davis2021} that, when they performed an in-depth
    analysis of \CSAM reported to \NCMEC\ during the fall of 2020, 90\% of the
    content were the same or visually similar as previously reported content.
    Further analysis of 150 accounts reported to \NCMEC\ during the summer of
    2020 and in January 2021 showed that 75\% of the people didn't exhibit
    malicious intent. Instead they shared \CSAM\ because of outrage, poor humor
    (``i.e., a child's genitals being bitten by an animal''), or sexting
    teenagers. These statistics suggest that at least Meta's piece counts are
    vastly exaggerating the problem and that additional, regular disclosures
    characterizing uniqueness and motivation are necessary.
\end{enumerate}

\NCMEC\ also releases report counts broken down by country. Since the largest
non-Chinese social media, with exception of Telegram, are all based in the
United States, that breakdown should be fairly representative of \CSAM\ sharing
patterns across the world.

Hence, to complete this investigation into the quality of transparency
disclosures, I decided to analyze \NCMEC's dataset for
2021~\cite{NcmecByCountry2021}. I did find several minor data quality issues.
Notably, the data included two different entries for French Guiana. Furthermore,
it included entries for (1) the Netherlands Antilles, (2) Bonaire, Sint
Eustatius, and Saba (3) Curaçao, as well as (4) Sint Maarten, even though the
former was split into the latter three in 2010. Finally, it included an entry
for Bouvet Island, which seems unlikely since the island, located in the South
Atlantic close to Antarctica, is an uninhabited nature preserve with no man-made
structures beyond a weather station.

To characterize geographic distribution, I determine regional totals as well as
per-capita rates per country. Taken together South and South East Asia are
responsible for 58.1\% of all reports, i.e., a clear majority. However, when
considering per-capita rates, the countries with the highest number of reports
per capita are concentrated on the Arab peninsula and in North Africa, abutting
the Mediterranean. The exception are the Cocos or Keeling Islands; at 168
reports per 596 capita, its rate is seven times higher than that of the second
ranked country, Libya. It likely also is a fluke thanks to the tiny population.
These results are consistent with those reported by a team from Google, \NCMEC,
and Thorn in 2019~\cite{BurszteinBrightea2019}. The supplemental materials for
this paper include both data and notebooks.



\subsection{What the Transparency Data Doesn't Cover}
\label{sec:census:limits}

It is imperative to keep in mind that social media's governance as reflected in
their transparency reports offer a partial picture only. Case in point is Meta
n\'ee Facebook. Its governance processes and transparency efforts may be sloppy
and suffer from data quality issues. But Meta also has made significant
investments into its governance by, for example, having external experts review
the metrics for its transparency reports~\cite{BradfordGriselea2019}, endowing
the external oversight board~\cite{BoteroMarinoGreeneea2020}, and having the
data collection audited~\cite{Sarang2022}. Yet the firm also has an astoundingly
deadly foreign record. By distributing and amplifying sectarian messaging, Meta
has directly contributed to genocides not once but twice! As a result, 25,000
Rohingya were murdered in Myanmar from August 2017 to August 2018 and 700,000
were driven into refugee camps across the border in Bangladesh, which have
become the densest settlement on
Earth~\cite{DeGuzman2022,HumanRightsCouncil2018}. As a result, between 385,000
and 800,000 Tigrayans were murdered in Ethiopia's northern-most
province~\cite{AnnysVandenBemptea2021,ChothiaBekit2022}, 882,000 people turned
into refugees, and 4.51 million people were internally displaced between
November 2020 and November 2022~\cite{UNICEF2023}.

In both countries, Meta put its profits well before any other considerations
including content moderation and safety. The firm keeps entering foreign
markets, not head but sales office first, even if user interfaces, policy
documents, help pages, content moderation tools, and machine learning models
have not yet been localized. In Myanmar, Facebook paid local cell phone
providers to provide their customers with free access through the Facebook app.
As a result, ``internet'' and ``Facebook'' are largely synonymous in the country
to this day~\cite{Strom2016}. At the same time, Facebook had no employees who
speak Burmese in early 2017. It ignored warnings from human rights organizations
and other \NGO{}s. It also got stymied by legacy encodings for Burmese
text~\cite{LaGrowPruzan2019,Wade2022}. When the firm finally got its act
together, it was too late: The genocide was well under
way~\cite{McLaughlin2018,MilkoOrtutay2022,Mozur2018,Ortutay2022}.

Over two years later, that same basic dynamic played out across Africa in
general and Ethiopia specifically. Facebook had opened its first office in
Africa, a \emph{sales} office, in 2015~\cite{Wagner2015}. When it started
pushing more aggressively into African countries in 2018, it publicly insisted
that it had learned its lessons~\cite{Tiku2018}. Yet it opened its first content
moderation office in (all of) Africa only in 2019~\cite{Agutu2019}. The firm
lacked an appreciation of the richness of dialects and languages spoken across
the continent~\cite{FickDave2019,JacksonTownsendea2022,Madung2021}. It also
exploited local workers~\cite{AlSibai2022,Perrigo2022,Perrigo2023}. Its own
Oversight Board sounded the alarm because the firm was ignoring telltale
signs~\cite{Faife2021}. When the firm finally got its act together, it was too
late: The genocide was well under
way~\cite{Allen2022,Gilbert2020,GlobalWitness2022,Ilori2020,Malik2022,ElliottChristopherea2021,ZelalemGuest2021,RobinsEarly2021}.

Since Meta is publicly traded, we might wonder what shareholders think about
this abysmal record. They don't seem to be too pleased either and did try
holding the firm's \CEO\ and Chairman of the Board Mark Zuckerberg accountable:
The annual meetings in 2018~\cite{Butler2018}, 2019~\cite{Sumagaysay2019},
2020~\cite{McRitchie2020}, 2021~\cite{Nix2021}, and 2022~\cite{WatersAgnew2022}
featured motions to strip him of one or both titles and the majority of shares
approved the motions in 2019, 2021, and 2022. But thanks to some funny business
with the allocation of votes per share, Mr Zuckerberg controls the majority of
votes. He also keeps voting for himself. So how much transparency and
accountability can we realistically expect from a genocide-enabling autocracy?
