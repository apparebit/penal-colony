% !TEX root = ../main.tex

In \emph{Discipline and Punish}, Foucault traces the transition from punishment
as a public and usually deadly spectacle to the modern prison and other
disciplinary institutions~\cite{Foucault1979}. He argues that this transition
did not happen for humanist concerns, as the result of reform efforts. Instead,
the driving force was the destabilizing impact of public executions. By being
rather ostentatious displays of power, they turned the criminal into sympathetic
victim. In contrast, executioner as well as sovereign became targets of popular
resentment. By simultaneously rationalizing, tempering, and distributing the
application of power, penal institutions avoid these downsides. They instead
instill discipline into the individual under their custody. As people
internalize discipline, that self-discipline obviates the need for more direct
applications of power and begets other disciplinary institutions, including
schools, hospitals, and factories. That way, we all turned into seasoned
practitioners of discipline.

The institutionalization of discipline does place constraints on the sovereign's
exercise of power and leads to an attendant loss of centralized control. That
last aspect is often lost on Foucault's readers. They are so razzle-dazzled by
Foucault's (admittedly fascinating) idealized disciplinary institution, the
panopticon, they don't realize that such a circular arrangement of cells around
a central monitoring station or tower simply can't scale beyond maybe 500 cells
--- at least in an architectural reality dominated by steel and concrete. Alas,
with the advent of cheap hardware sensors and powerful machine learning models,
re-centralizing control through ubiquitous digital surveillance is becoming
practical. While East Germany did manage just that for 40 years, its lo-tech
approach to central control was too expensive to be sustainable in the long run.
In one district, 18\% of the population were active informants for the Stasi,
that country's vicious state security service~\cite{Kellerhoff2022}.

In contrast, China under its current, particularly authoritarian president Xi
Jinping --- or more concisely Xina --- is reaping the benefits of readily
available sensors and algorithms. By rolling out ever more intrusive yet
centralized control, Xina is erasing the distinction between prison and
not-prison at scale~\cite{Grauer2021,MozurXiaoea2022,SmithIV2016}. Ironically,
some of that is driven by American innovations on predictive
policing~\cite{PerryMcInnisea2013,SmithIV2016,Sprick2019}. But the excessive
intrusiveness of the Han n\'ee Borg declaring ``Resistance is futile. You will
be assimilated!'' also makes Xina's \emph{surveillance state} an unsuitable
model for algorithmic control in western democracies.

The \emph{carceral state} in the United States comes closer~\cite{Simon2007}. It
has the largest number of prisoners in the world — as of 2022, the country
accounted for 4.25\% of the world population~\cite{Worldometer2023} and 20\% of
people in prisons or jails across the world~\cite{SawyerWagner2022} — and the
highest incarceration rate in the world — at 625 per 100,000 in 2019, the US
imprisoned 6.0$\mspace{1mu}\times$ as many people as Canada,
9.4$\mspace{1mu}\times$ as Germany, and 17.5$\mspace{1mu}\times$ as
Japan~\cite{WorldPrisonBrief2023}. Additionally, roughly double as many people
are under the control of the carceral state through probation or parole and may
be locked up at a moment's notice for largely arbitrary
reasons~\cite{SawyerWagner2022}. Next, many of the imprisoned are not there
because they have been convicted of a crime: Half of those imprisoned in local
jails (as opposed to federal prisons) are there for pretrial
detention~\cite{SawyerWagner2022}.

Worse, mass incarceration disproportionally impacts poor and minority
populations. Notably, Black Americans made up 12\% of the US adult population in
2018 but also made up 33\% of all people serving sentences greater than one year
in state or federal prison~\cite{Gramlich2020}. Incredibly, that is after a 34\%
reduction of their incarceration rate since 2006 and comparatively smaller
reductions for other ethnicities. They also tend to be locked up, far from home,
in districts that are far more white than their homes, cutting them off from
family and friends~\cite{WagnerKopf2015}.

As I'll discuss in more detail in Section~\ref{sec:census}, some of the more
extreme and hence also white supremacist practices of the American carceral
state, notably the (figurative) death penalty and three strikes rules, have
direct equivalents in social media content moderation. In the opposite
direction, the carceral state is an early and aggressive adopter of algorithmic
enforcement
technologies~\cite{AngwinLarsonea2016,EPIC2020,Hao2019,ReddenODonovanDixea2020,Yampolskiy2016}.
At the same time, algorithmic enforcement clearly isn't limited to the
government. Corporations and universities are deploying punitive interventions
just as well outside the criminal injustice system. In doing so, they may even
innovate on punishment. As I'll show in Sec.~\ref{sec:tweet:da:fe}, Twitter's
enforcement process harkens back to punishment as a performance, but it does so
while (ingeniously) avoiding the destabilizing public spectacle. In short, this
paper is concerned with algorithmic interventions outside of the immediate
control of the sovereign state. That also distinguishes this work from previous
papers, which point towards the punitive potential but do so in the context of
the surveillance and carceral
states~\cite{DehlendorfGerety2021,McElroyWhittakerea2021}.

I am proposing the \emph{penal colony} as closest historical precedent and as
fitting model for contemporary algorithmic practices outside the criminal
injustice system. The French version of \emph{transportation} --- the practice
of sending prisoners to far locales --- is far more recent than we'd probably
like to acknowledge. France began turning French Guiana into one large penal
colony from 1852 onwards --- after the British had already begun unwinding
theirs --- and closed the colony only in
1953~\cite{Aldrich2010,Anderson2018,Spierenburg2009}. For the last 70 years or
so, transportation was reserved for convicts sentenced under France's own three
strikes laws. It also was almost always terminal: Only 2,000 out of 70,000
prisoners returned to France during their
lifetimes~\cite{WallechinskyWallace1978}. For that reason, prisoners referred to
the penal colony as ``dry guillotine''~\cite{Furlong1913,ReneBelbenoit1938}. Yet
discipline was surprisingly inconsistent, even lax, depending on location.

Foucault had surprisingly little to say about the penal
colony~\cite{Redfield2005}, even though transportation must be understood as a
distinct intermediate stage in penal history. As such, it combines aspects from
the performance of punishment and the discipline of prisons. Notably, like
earlier practices, transportation is usually terminal. But unlike earlier
practices, the penal colony is discretely out of sight. The penal colony also
incorporates a disciplinary component, typically involving hands-on labor to
create the infrastructure for more general colonization. The \emph{stochastic
penal colony} preserves that distinctness and stands apart from both carceral
and surveillance states. Its excessiveness does remind of the American carceral
state and its technology is much the same as Xina's surveillance state. But
there also is no central control or even intent. And while its downsides,
predictably, worsen along racial lines, it ensnares the privileged, including
many white people, almost as easily.

The remoteness of the penal colony, both literally and figuratively, also turns
it into an effective investigative device that renders contemporary practice
strange again and hence amenable to analysis. While that renders the stochastic
penal colony a largely ahistorical concept, its intellectual lineage does trace
right back to the former penal colony in French Guiana: The stochastic penal
colony obviously draws on Franz Kafka's 1919 short story \emph{In the Penal
Colony}~\cite{Kafka1995}. Kafka, in turn, was influenced~\cite{Robertson2017} by
Octave Mirbeau's 1899 novel \emph{The Torture Garden}~\cite{Mirbeau2008}. While
taking place in an imaginary China, its year of publication and dedication ---
``To Priests, Soldiers, Judges / to mean who rear, lead, or govern men / I
dedicate these pages of murder and blood.'' --- point to the Dreyfus affair as
primary inspiration. Alfred Dreyfus, a Jew and French military officer, had been
falsely convicted for espionage in early 1895 and again in 1899 --- with rampant
antisemitism leading to the systematic suppression of exculpatory evidence and
complete disregard of the real spy's public confession in 1898. As a result, Mr
Dreyfus spent 1895--1899 on Devil's Island off the coast of French Guiana, a
particularly harsh location in the network. Coincidentally, the Dreyfus affair
also popularized the word ``intellectual,'' albeit starting out as a
pejorative~\cite{Drake2005,StudentsAtTheUniversityOfBristol2021}.

Besides, the stochastic penal colony~\emo{desertisland} provides an excellent
home for the current pandemonium of stochastic
parrots~\emo{parrot}~\cite{BenderGebruea2021}!
