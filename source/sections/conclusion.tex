\section{Conclusion}
\label{sec:conclusion}

In this paper, I introduced the stochastic penal colony as the unfortunate
reality of many practical algorithmic interventions. As my case studies on
\DALLE's and Twitter's enforcement illustrate, these algorithmic interventions
vary considerably in how they mete out punishment. OpenAI's maximally punitive
design apparently was too much even for its creators, who considerably softened
the blows since the original roll out. In contrast, Twitter manages to go from
best-practices policy to an innovative combination of inquisition and inhuman
process. My social media census makes clear that the two are by no means
exceptions. The stochastic penal colony certainly is well established. Finally,
my probing of \DALLE's enforcer illustrates the limitations of algorithmic
content moderation. It simply ain't safe, especially against a motivated
adversary.

So what can we do about the stochastic penal colony? I don't have any good
answers (yet), seeing that this research is only at its beginning and much
follow-up work is required, including quantitative inquiries that hopefully
support my qualitative observations. There is one thing, however, that most
certainly won't make a difference: More \V{AI} ethics statements! None of the
existing ones are worth the electrons being pumped through circuits to render
them on screen. They are so removed from practical reality, they might as well
not exist~\cite{Hagendorff2022,Munn2022,WhittlestoneNyrupea2019}.

The one scientific discipline with a practicable ethics, medicine, gets by with
just one principle: \emph{Primum non nocere}! First, do no harm! Alas, when it
comes to algorithmic interventions, it's far too late for doing \emph{no} harm.
Instead, I am proposing harm reduction as an admittedly weaker \V{AI} ethics, but
hopefully one that can serve as practical countermeasure to the stochastic penal
colony. As a gay man of a certain age, I saw the devastation of \textsc{aids}
first-hand and also know that harm reduction through safer sex has saved
countless lives, including my own. Harm reduction would have saved thousands of
opioid addicts from lethal overdoses in the United States. It's easy enough to
understand~\cite{HRI2020,HRI2022,Interlandi2023,MarlattLarimerea2011,OpenSocietyFoundations2021}.
It's effective. Anyone involved in the design or implementation of algorithmic
interventions should familiarize themselves with the idea. As the example of
devolution in \S\ref{sec:devolution} illustrates, it might just make a
real difference!
