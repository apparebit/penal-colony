% !TEX root = ../main.tex

\section{Tweet-Da-F\'e}
\label{sec:tweet:da:fe}

Early one morning in October 2021, I had just finished reading an article about
some oil industry association spending millions of dollars on lobbying and
advertising to derail the Biden administration's push for climate change
legislation~\cite{Tabuchi2021}. Additionally, three of the association's larger
member companies spent millions of dollars each towards that same goal ---
despite also being responsible for 8.7\% of all global CO$_2$ emissions since
1965~\cite{TaylorWatts2019}.

I was enraged. To vent, I composed a caustic tweet that @-mentioned the three
firms and stated that I was looking forward to their \CEO{}s facing capital
punishment for genocide. I was aware of the statement's severity and hence
incivility while writing the tweet. But I told myself that it was ok, since the
statement implies a formal, legal process that still is practiced in the United
States. I even included that argument in both of my appeals.

I remain somewhat ambivalent about the tweet. When ExxonMobil's internal
projections from 1977 to 2003 ``accurately forecast warming that is consistent
with subsequent observations''~\cite{SupranRahmstorf2023}, when birds fall dead
from the skies~\cite{Dave2022}, 11 billion crabs just
vanish~\cite{Olmstead2022}, and a third of Pakistan floods~\cite{Chughtai2022}
because of that warmining, it is hard \emph{not} to wish harm on major oil
companies and their \CEO{}s. Yet, the record-setting execution spree towards the
end of Donald Trump's
presidency~\cite{Arnsdorf2020,Kovarsky2022,SuebsaengReis2023} also makes clear
that the content of the tweet isn't just utterly incompatible with this paper's
basic premise, but tweet and oil company greed share the same basic inhumanity.
It appears then that I acted, more or less, as a \emph{daily active shithead}
that morning~\cite{Sherman2021}.

In either case, Twitter's \AI{} was not impressed by the tweet's incivility.
Within a couple of seconds after posting at most, it removed the tweet and
locked my account. Compared to the content warning for \DALLE, the stated
justification was far more specific:

\begin{quote}
\openfat\textbf{Violating our rules against
\href{https://web.archive.org/web/20220905021323/https://help.twitter.com/en/rules-and-policies/abusive-behavior}{abuse
and harassment}.}

You may not engage in the targeted harassment of someone, or incite other people
to do so. This includes wishing or hoping that someone experiences physical
harm.\closefat{}
\end{quote}

\noindent{}The linked policy on abusive behavior, reproduced in
Appendix~\ref{sec:twitter:abusivebehavior} on
page~\pageref{sec:twitter:abusivebehavior}, is not only specific but genuinely
helpful. It is written in accessible, well-structured prose: The policy starts
with its rationale, is followed by the different kinds of abusive content, and
concludes with a range of possible sanctions. The mid-section on kinds of
abusive content features well-delineated and reasonable prohibitions. It even
reassures readers that the firm is well aware that some tweets, by themselves,
may appear to violate the policy but, when considered in their original context,
do not.

Thanks to the effective presentation, finding the concrete prohibition
applicable to my tweet was easy: ``Wishing, hoping, or calling for serious harm
on a person or group of people.'' After elaborating on possible context and
giving examples, the policy --- rather reasonably --- allows that some wishes of
harm may be justified, in the heat of the moment, as expressions of outrage. In
such cases, Twitter still requires offending tweets to be deleted but does not
impose penalties. Apparently, rapists and child abusers count as legitimate
targets under this exception, but oil company \CEO{}s do not --- yet.

I appealed the decision by Twitter's AI that same morning. Or at least, I tried
to: Twitter's form for filing an appeal seems to have the same character limit
as a tweet. That excludes most arguments besides a succinctly stated single
reason. Alas, my justification was far from that and, not surprisingly, Twitter
rejected the appeal three days later. However, the form email notifying me of
the rejection wasn't even filled in, despite containing instructions in \HTML\
comments. Since I had located another page for launching an appeal that wasn't
marred by the original form's character limit, I tried again with that form,
this time focusing mostly on the bad form. When that second appeal went
unanswered for three weeks, I gave up. I withdrew my appeal, acknowledged that I
``violated the Twitter Rules,'' and deleted the offending tweet --- all with one
click on the red ``Delete'' button.

Alas, residual effects from the episode remain. When I try to sign up to Twitter
for Professionals, I get a notification that ``something's missing,'' even
though my account meets all criteria stated in Twitter's documentation. Yet a
satirical account of mine, which I opened more recently and which describes my
alter ego as a ``lifelong practitioner of faggotry, promoter of the gay agenda,
and unrepentant socialist monarchist,'' could sign up to Twitter for
Professionals within days of account creation.


\subsection{A Punishing Performance}

OpenAI's policy enforcement is consistently Kafkaesque. As a result, it slowly
and methodologically delivers \lingchi, death by a thousand cuts, through the
algorithm's Harrow. In contrast, Twitter's process is far more uneven. In fact,
as discussed above, the public-facing policy is rather reasonable and reasonably
documented. There are no obvious Kafkaesque co-factors (though there is a deeply
problematic omission I'll discuss in the next subsection below). But that all
changes the moment a policy violator has been identified and can serve as
Condemned in Twitter's punishing performance.

The stage for that play is the Condemned's Twitter, which as illustrated in
Appendix~\ref{sec:twitter:thetweet} on page~\pageref{sec:twitter:thetweet} has
been reduced to the violative tweet, i.e., the one tweet no one else can see.
The set design may be minimalist, but it also is effective: It keeps reminding
the audience of the transgression that triggered this performance. Likewise, it
reminds the audience that the final arbiter of account access is Twitter --- and
Twitter only. Speaking of audience, this performance departs from more
traditional theatrical practice by making the Condemned the entirety of both
audience and cast.

To keep things interesting, the play requires that audience and cast make one
significant choice, namely the duration of the performance. Options include
hours (Condemned skips appeal), days (Condemned appeals), or forever (Condemned
walks out). I use the word ``appeal'' in the parenthetical elaborations because
it is concise and clear. But it is unclear how meaningful appeals actually are
when Twitter limits justifications to 280 characters, keeps admonishing the
Condemned that ``you won't be able to access your Twitter account'' and to
``just delete your content,'' provides no explanation when rejecting an appeal,
and discloses no statistics on appeals in its yearly transparency report (more
on that in Sec.~\ref{sec:census}). Meanwhile, Twitter's original explanation
that, ``while in this state, you can still browse Twitter, but you're limited to
only sending Direct Messages to your followers --- no Tweets, Retweets, Fleets,
follows, or likes'' is just plain false. While the Condemned is logged in, the
punishing performance continues.

While the contours of the set are static, incidental text around the one tweet
that no one else can see does change and thereby accommodates a climax of sorts:
To end the punishing performance, the Condemned must acknowledge their guilt and
then ``delete'' the one tweet no one else can see. That tweet, of course, is a
digital artifact that only ever existed on Twitter, always was under full
control by Twitter, and has already been censored by Twitter for everyone else.
It only exists for this punishing performance, so that the Condemned can submit
to the mighty Twitter by deleting, purging, vanquishing the one tweet no one
else can see. That makes the entire performance a farce, a coercive, punitive,
and slightly degrading farce. There's four Kafkaesque co-factors right there.

The fifth Kafkaesque co-factor is that the punishing performance certainly is no
accident or happy coincidence. It obviously is carefully designed through and
through. Hence, its maximizing the emotional impact of the final submission also
is designed and intentional. Yet its mechanized, individualized implementation
also ensures that not too many people will ever notice. After all, the vast
majority of Twitter users will never witness, let alone experience this
performance first hand. If they need to, they can always reassure themselves by
consulting Twitter's oh so reasonable publicly posted policies. Should they find
themselves confronted with a hurt or angry or overly critical Condemned
complaining about systemic corporate abuse, that Condemned is bound to come
across as a bit unreasonable, emotional, or hysterical — and hence so much
easier to dismiss and ignore. Besides, they did something wrong, didn't they!?

Overall, as a display of naked power, the punishing performance fits right in
with Catholic Inquisition and Maoist denunciation rallies. Yet Twitter's
enforcement ritual does not require physical torture, ``only'' a bit of
emotional abuse. That difference may just be the result of the electronic medium
not allowing for more. But that medium also is the critical enabling factor: It
enables nano-targeting the punishing \emph{performances} to their individualized
audiences and casts, while also avoiding the destabilizing impact of the
\emph{public} punishment performances identified by Foucault.

The reason that Condemned put up with this shit was pre-Musk Twitter's rather
unique position as breaking news service, political townsquare, professional
society, and corporate customer service platform in one. Thanks to that
combination, the threat of persistent account lockout was significant and,
depending on a user's Twitter presence and follower counts, could approach
something like social death. But thanks to Mr Musk's unique leadership since
taking over the firm, Twitter lost many users as well as advertisers, and hence
is much diminished in status.


\subsection{Twitter's Neutrality and Trustworthiness}

To start with, an organization that turns punishment into a theatrical
performance just isn't particularly trustworthy. Especially one that, as I
suspect, deliberately created this intervention with all its performative and
punitive excess. The reason for my suspicion is that content policy, enforcement
process, and transparency reports all share a misdirection that also is easy to
miss because it is by omission: They not even mention algorithms --- despite
their critical day-to-day role since the start of the pandemic at the very
least. Given the careful, systematic consideration reflected throughout
Twitter's public documentation, it's hard to believe that this omission wasn't
intentional. The firm certainly had plenty of time to update its documentation
since.

\begin{table}
\caption{Search terms and number of hits on Twitter's help pages (21 October, 2022)}
\label{table:search}
\begin{tabular}{lr}
\textbf{Search Term} & \textbf{Results} \B \\ \hline
AI & \T0 \\
algorithm & 5 \\
artificial intelligence & 1 \\
machine learning & 3 \\
\end{tabular}
\end{table}

Furthermore, that omission isn't limited to content policy and enforcement but
extends to \emph{all} of Twitter's help pages. Table~\ref{table:search}
quantifies the results from searching for common variations of the term ``\AI''
using Twitter's own search functionality. The darth of material is striking: Not
only were there hardly any mentions, but existing ones amounted to little more
than an acknowledgement that, for instance, top tweets, topics, and
recommendations are curated by algorithms. There certainly were no
context-providing model cards or system cards to be found.

Twitter's transparency report (which the new management removed mid-December
2022) nonetheless helps clarify an important aspect of its automated content
review, namely its exact timing. When my account was blocked in October 2021,
the notification thereof was nearly instantaneous after posting, but I wasn't
sure whether Twitter's application had confirmed posting my tweet. The
difference here is whether Twitter reviews tweets before posting them or in
parallel to posting them. The former may slow down posting somewhat but ensures
that all policy violations are caught before posting. The latter optimizes for
the common case, speedily posting compliant tweets, but does result in tweets
that violate Twitter's policies being visible for some amount of time.

The transparency report indirectly confirms that Twitter is posting and
reviewing tweets in parallel. Similar to Pinterest and YouTube in their
transparency reports, it quantifies the number of impressions before content was
removed. But where Pinterest uses the bands 0, 1–9, 10–100, and >100 and YouTube
uses 0, 1–10, and >100, Twitter uses <100, 100–1,000, >1,000. The omission of 0
as a band by itself is curious, since it covers content that was never
accessible. But if we assume that each firm picks bands to look its best, the
sizing of the first band at <100 impressions strongly suggests that Twitter’s
automated review happens in parallel to posting and tends to be slower than
Printerest’s and YouTube’s.

By comparison, user-initiated content review moves at a much slower pace. When I
reported a tweet by a news organization that seemed to wish a harsher fate on
the Parkland school shooter after his sentencing to life in prison, it took me a
handful of smartphone screens to complete the report. Twitter’s response in turn
took approximately eight hours. Not surprisingly, that wish of harm was found
acceptable.

Elon Musk's reign of public insecurities and summary firings has one benefit for
the purposes of this paper: Him granting access to internal communications
including about content moderation. To right wingers, these so-called Twitter
files are revalations of Machiavellian scheming that was hitherto unimaginable.
To left wingers, uhm, Twitter files? What's that? When they have heard about
them, they have several objections: Bari Weiss is a little too fond of
superlatives. Yes, but she also is launching her own news organization
(imaginatively named ``The Free Press''), so I am inclined to ignore the noise.
Matt Taibbi doxing Twitter employees and Ella Irwin, the firm's new head of
trust, casually showing off her goddess console, which gives her access to any
and all content including private messages, are deeply problematic. But the
latter reflects badly on pre-Musk Twitter, too. They created the console.

More importantly, reports on the handling of blacklists suggests that the lack
of transparency has also resulted in abuse of the powers. Reporting on the
handling of misinformation raises grave concerns about a category that just
invites Orwellian abuse. And finally, the many ways Twitter coddled both left
and right pressure groups shows that us regular people were second class on
Twitter. So no, Twitter isn't neutral. Twitter isn't trustworthy.



For once, the truth does
indeed lie in the middle. Yes, Bari Weiss is a little too fond of superlatives.
But she is also launching her own news website, so I'm inclined to overlook
that. Yes, Matt Taibbi doxing former employees was in spectacularly bad form, no
argument there. And yes, Twitter's new head of trust taking pictures of her
goddess console that gives her access to any and all



Him granting access to several journalists to internal data on
content moderation has conclusive demonstrated that Twitter's teams abused their
powers, with secret blacklists and heavy-handed enforcement of its
misinformation rules. I am full aware that these so-called ``Twitter files'' are
the subject of partisan dissonance, with right leaning folk including the
journalists writing the articles comparing them to Watergate and whatever,
whereas left leaning folk and their newspapers including New York Times declared
them a complete non-event and now just ignore them. In this case, the truth is
somewhere in the middle. Yes, Matt Tahibi doxing former Twitter employees was in
spectacularly bad form --- and also a violation of Twitter's community rules.
Yes, Bari Weiss is a bit too fond of superlatives. But she also is a launching a
new digital news organization with just those reports, so I am inclined to
overlook the noise.
