% !TEX root = ../main.tex

\section{Tweet-Da-F\'e: A Punishing Performance}
\label{sec:tweet:da:fe}

Early one morning in October 2021, I had just finished reading an article about
some oil industry association spending millions of dollars on lobbying and
advertising to derail the Biden administration's push for climate change
legislation~\cite{Tabuchi2021}. Additionally, three of the association's larger
member companies spent millions of dollars each towards that same goal ---
despite also being responsible for 8.7\% of all global CO$_2$ emissions since
1965~\cite{TaylorWatts2019}.

At the time, I was enraged. To vent, I composed a caustic tweet that @-mentioned
the three firms and stated that I was looking forward to their \CEO{}s facing
capital punishment for genocide. I had some awareness of the statement's
severity and hence incivility while writing the tweet. But I told myself that it
was ok, since the statement implies a formal, legal process that still is
practiced in the United States. In context of a paper that takes a principled
stand against carceral state and stochastic penal colony, that line of reasoning
seems specious at best. Yet I also used just that line of reasoning in at least
one of two appeals. It appears that I am all too capable of acting as
\emph{daily active shithead}~\cite{Sherman2021}.

Twitter's \AI{} was not impressed by the tweet's arguable legality. It
intercepted the tweet and also locked my account. Compared to \DALLE, the stated
justification was far more specific:
\begin{quote}
\openfat\textbf{Violating our rules against abuse and harassment.}

You may not engage in the targeted harassment of someone, or incite other people
to do so. This includes wishing or hoping that someone experiences physical
harm.\closefat{}
\end{quote}

\noindent{}The linked policy on abusive behavior (reproduced in
Appendix~\ref{adx:twitter:abusive-behavior} on
page~\pageref{adx:twitter:abusive-behavior}) not only is specific but also
genuinely helpful. It is well-structured, starting with the rationale, followed
by the different kinds of abusive content, and concluding with the range of
possible sanctions. The mid-section on kinds of abusive content has a reasonable
scope, with well-delineated prohibitions. It even reassures readers that the
firm is well aware that some tweets, by themselves, may appear to violate the
policy but, when considered in their original context, do not. Finally, the
policy is well-written, in prose, instead of listing just keywords. In short,
the policy is presented so that users can follow along.

Thanks to the effective presentation, finding the concrete prohibition
applicable to my tweet was easy: ``Wishing, hoping, or calling for serious harm
on a person or group of people.'' After elaborating on possible context and
giving examples, the policy --- rather reasonably --- allows that some wishes of
harm may be justified, in the moment, as expressions of outrage. In such cases,
Twitter still requires offending tweets to be deleted but does not impose
penalties. Apparently, rapists and child abusers count as legitimate targets
under this exception, but oil company \CEO{}s, who bear significant
responsibility for bringing vast regions of Earth to the brink of becoming
inhabitable, do not --- yet.

I appealed the decision by Twitter's AI that same morning. Or at least, I tried
to: Twitter limits the justification for an appeal to 280 characters, just as it
does for tweets. That prevents most arguments beyond a Bart-Simpsonian
insistence that ``I didn't do it!'' Not surprisingly, Twitter rejected the
appeal three days later. Somewhat surprisingly, the form email notifying me of
the rejection wasn't even filled in — despite a \HTML{} comment to ``add
bulleted policy list after running script.'' In the meantime, I had located
another page for launching an appeal that was not marred by the character limit
and so I submitted a more fully developed but still succinct justification. When
that second appeal went unanswered for three weeks, I gave up. I withdrew my
appeal, acknowledged that I ``violated the Twitter Rules,'' and deleted the
offending tweet --- all with one click on the red ``Delete'' button.

Alas, residual effects from the episode remain. When I try to sign up to Twitter
for Professionals, I get a notification that ``something's missing,'' even
though my account meets all criteria stated in Twitter's documentation. Yet a
satirical account of mine, which I opened more recently and which describes my
alter ego as a ``lifelong practitioner of faggotry, promoter of the gay agenda,
and unrepentant socialist monarchist,'' could sign up to Twitter for
Professionals within days of account creation.


\subsection{The Performance of Punishment}

Whereas OpenAI's policy enforcement is consistently Kafkaesque and slowly and
methodologically delivers
\raisebox{-0.1ex}{\includegraphics[height=0.9em]{emoji/lingchi}}, death by a thousand
cuts, through the algorithm's Harrow, Twitter's process is far more uneven and,
in fact, very much bimodal. As discussed above, the public-facing side, i.e.,
Twitter's policy, is rather reasonable and reasonably documented. There are no
Kafkaesque co-factors. But the moment a policy violator has been identified,
Twitter's enforcement process turns the very opposite of reasonable and treats
the account holder as Condemned in a rather tedious morality play.

The stage for that play is the Condemned's Twitter, which has been reduced to
the one tweet no one else can see. The set design may be minimalist, but it also
is amazingly effective: It constantly reminds the audience of the transgression
that triggered this performance. Likewise, it reminds the audience that the
final arbiter of account access is Twitter — and Twitter only.

Speaking of audience, this performance departs from more traditional theatrical
practice by making the Condemned the entirety of both audience and cast. Even
more unusually, the play, as written, requires that the Condemned makes an
apparently substantial choice, namely whether to file an appeal. In turn, that
choice determines whether the performance lasts for hours (Condemned skips
appeal), days (Condemned appeals), or forever (Condemned walks out).

For dramatic effect, the play makes liberal use of misdirection. That starts by
calling the Condemned's only choice an ``appeal.'' When Twitter limits the form
to file said appeal to 280 characters, keeps admonishing the Condemned that
``you won't be able to access your Twitter account'' and to ``just delete your
content,'' provides no explanation when rejecting an appeal, and discloses no
statistics on appeals in its yearly transparency report, it's clear enough that
there is no meaningful appeal to be had and that this option largely exists to
provide the illusion of due process.

In either case, the play builds up to its final scene, which has the Condemned
acknowledge their moral failing and then “delete” the tweet no one else can see.
That tweet, of course, is a digital artifact that only ever existed on Twitter,
always was under full control by Twitter, and had already been censored by
Twitter. In short, the label is another misdirection that serves to put the
spotlight on the Condemned while Twitter runs the show.

It doesn't take much to recognize the farcical aspects of this performance. But
the very obviousness of the play's final charade also maximizes its degrading
emotional impact. After all, the play is staged, under duress, by a cast of one
for an audience of one — who are the same one. The punitive account lockout and
the symbolic submission to Twitter's demand for deletion very much enhance the
emotional payoff of the coerced admission of guilt: Feelings of being
manipulated, belittled, demeaned.

Twitter's carefully compartmentalized policy enforcement is the primary reason
the firm can get away with this. The vast majority of Twitter users will never
witness, let alone experience this morality play first hand. If they need to,
they can always reassure themselves by consulting Twitter's oh so reasonable
publicly posted policies. Should they find themselves confronted with a hurt or
angry Condemned complaining about corporate abuse, that Condemned is bound to
come across as a bit unreasonable, emotional, or hysterical — and hence so much
easier to dismiss and ignore.

As far as performative displays of naked power for maintaining social control
are concerned, Twitter's enforcement process belongs to the same tradition as
Catholic Inquisition and Maoist denunciation rallies. Yet two critical aspects
ensure that it represents a significant departure from historical precedent that
stands on its own. First, Twitter's enforcement process does not require
physical torture, ``only'' emotional abuse. Very much like inquisition and
denunciation rallies, that abuse is normalized through an unshakable belief in
the Condemned's guilt — and one's own moral supremacy. Second, it dispenses with
the public spectacle and instead microtargets each performance to a carefully
selected individual serving as sole performer and also audience of one.

These radical refinements of inquisitory prosecutions and denunciation rallies
are of course enabled by information technology, notably \AI. However, I very
much doubt that a similar enforcement process could be created by companies
other than, possibly, Meta's Facebook or Instagram divisions even if they tried.
Pre-Musk Twitter functioned as breaking news service, political townsquare,
professional society, and corporate customer service platform in one. Hence much
of the activity in the human social networks for many Twitter users may just
happen on that technological social network. That turns a persistent account
lockout into a threat akin to actual social death.

The severity of this threat varies with user and probably is most acute for
influencers, i.e., users who are deeply invested in the platform and have a
large number of followers only thanks to the platform. Though the way Donald
Trump stopped dominating news headlines following his deplatforming by Twitter
in the wake of his attempted coup on January 6, 2021 illustrates that the social
network can provide significant amplification even for the \textsc{us}
President. Reducing Twitter to the one tweet no one else can see avoids putting
the threat of losing access into easily quotable text while still making it
palpable. Thereby, it cleverly reduces reputational risk to the firm while also
making (most) users sufficiently pliant to accept the abuse by the enforcement
process!


\subsection{Twitter's Neutrality and Trustworthiness}

To start with, an organization that turns punishment into a theatrical
performance just isn't particularly trustworthy. Especially one that, as I
suspect, deliberately created this intervention with all its performative and
punitive excess. The reason for my suspicion is that content policy, enforcement
process, and transparency reports all share a misdirection that also is easy to
miss because it is by omission: They not even mention algorithms --- despite
their critical day-to-day role since the start of the pandemic at the very
least. Given the careful, systematic consideration reflected throughout
Twitter's public documentation, it's hard to believe that this omission wasn't
intentional. The firm certainly had plenty of time to update its documentation
since.

\begin{table}
\caption{Search terms and number of hits on Twitter's help pages (21 October, 2022)}
\label{table:search}
\begin{tabular}{lr}
\textbf{Search Term} & \textbf{Results} \B \\ \hline
AI & \T0 \\
algorithm & 5 \\
artificial intelligence & 1 \\
machine learning & 3 \\
\end{tabular}
\end{table}

Furthermore, that omission isn't limited to content policy and enforcement but
extends to \emph{all} of Twitter's help pages. Table~\ref{table:search}
quantifies the results from searching for common variations of the term ``\AI''
using Twitter's own search functionality. The darth of material is striking: Not
only were there hardly any mentions, but existing ones amounted to little more
than an acknowledgement that, for instance, top tweets, topics, and
recommendations are curated by algorithms. There certainly were no
context-providing model cards or system cards to be found.

Twitter's transparency report (which the new management removed mid-December
2022) nonetheless helps clarify an important aspect of its automated content
review, namely its exact timing. When my account was blocked in October 2021,
the notification thereof was nearly instantaneous after posting, but I wasn't
sure whether Twitter's application had confirmed posting my tweet. The
difference here is whether Twitter reviews tweets before posting them or in
parallel to posting them. The former may slow down posting somewhat but ensures
that all policy violations are caught before posting. The latter optimizes for
the common case, speedily posting compliant tweets, but does result in tweets
that violate Twitter's policies being visible for some amount of time.

The transparency report indirectly confirms that Twitter is posting and
reviewing tweets in parallel. Similar to Pinterest and YouTube in their
transparency reports, it quantifies the number of impressions before content was
removed. But where Pinterest uses the bands 0, 1–9, 10–100, and >100 and YouTube
uses 0, 1–10, and >100, Twitter uses <100, 100–1,000, >1,000. The omission of 0
as a band by itself is curious, since it covers content that was never
accessible. But if we assume that each firm picks bands to look its best, the
sizing of the first band at <100 impressions strongly suggests that Twitter’s
automated review happens in parallel to posting and tends to be slower than
Printerest’s and YouTube’s.

By comparison, user-initiated content review moves at a much slower pace. When I
reported a tweet by a news organization that seemed to wish a harsher fate on
the Parkland school shooter after his sentencing to life in prison, it took me a
handful of smartphone screens to complete the report. Twitter’s response in turn
took approximately eight hours. Not surprisingly, that wish of harm was found
acceptable.

Elon Musk's reign of public insecurities and summary firings has one benefit for
the purposes of this paper: Him granting access to internal communications
including about content moderation. To right wingers, these so-called Twitter
files are revalations of Machiavellian scheming that was hitherto unimaginable.
To left wingers, uhm, Twitter files? What's that? When they have heard about
them, they have several objections: Bari Weiss is a little too fond of
superlatives. Yes, but she also is launching her own news organization
(imaginatively named ``The Free Press''), so I am inclined to ignore the noise.
Matt Taibbi doxing Twitter employees and Ella Irwin, the firm's new head of
trust, casually showing off her goddess console, which gives her access to any
and all content including private messages, are deeply problematic. But the
latter reflects badly on pre-Musk Twitter, too. They created the console.

More importantly, reports on the handling of blacklists suggests that the lack
of transparency has also resulted in abuse of the powers. Reporting on the
handling of misinformation raises grave concerns about a category that just
invites Orwellian abuse. And finally, the many ways Twitter coddled both left
and right pressure groups shows that us regular people were second class on
Twitter. So no, Twitter isn't neutral. Twitter isn't trustworthy.



For once, the truth does
indeed lie in the middle. Yes, Bari Weiss is a little too fond of superlatives.
But she is also launching her own news website, so I'm inclined to overlook
that. Yes, Matt Taibbi doxing former employees was in spectacularly bad form, no
argument there. And yes, Twitter's new head of trust taking pictures of her
goddess console that gives her access to any and all



Him granting access to several journalists to internal data on
content moderation has conclusive demonstrated that Twitter's teams abused their
powers, with secret blacklists and heavy-handed enforcement of its
misinformation rules. I am full aware that these so-called ``Twitter files'' are
the subject of partisan dissonance, with right leaning folk including the
journalists writing the articles comparing them to Watergate and whatever,
whereas left leaning folk and their newspapers including New York Times declared
them a complete non-event and now just ignore them. In this case, the truth is
somewhere in the middle. Yes, Matt Tahibi doxing former Twitter employees was in
spectacularly bad form --- and also a violation of Twitter's community rules.
Yes, Bari Weiss is a bit too fond of superlatives. But she also is a launching a
new digital news organization with just those reports, so I am inclined to
overlook the noise.
