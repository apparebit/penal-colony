% !TEX root = ../main.tex

%\newpage
\section{Research Ethics}
\label{sec:research-ethics}

Somewhat unusually, I conducted the research for this paper and wrote the paper
without institutional support. That included not having access to an
institutional library. Thankfully, regular search engines do a passable job at
surfacing academic literature. Furthermore, preprint archives and open access
publications make many publications of the last decade or so readily available.
However, that still leaves a large number of publications behind the paywalls of
academic publishers.

I observed per-article rates from \$15 by professional societies to around \$50
(or more) by for-profit publishers, which strike me as excessive. While
publishers often offer better rates for bulk access, they still are entirely
unreasonable given that publishers paid nothing for the hard work of writing,
reviewing, and editing these articles. Instead, I largely relied on Sci-Hub to
access paywalled research articles. Since my prior academic publications did not
benefit from open access options, I encourage others to also rely on Sci-Hub for
accessing them.

The paper itself combines two auto-ethnographic case studies with a survey based
on transparency reports and experiments that actively probed the beta release of
a production system. Sec.~\ref{sec:census}, the section presenting the survey,
already covers an attempt at validating some of the underlying transparency
disclosures. It also includes a discussion of the limitations inherent in such a
survey, especially when one of the covered organizations effectively is beholden
to a single person. Consequently, there is little more to say about the survey.

Both research and auto-ethnographic case studies are initially based on my
personal interests. Given the lack of institutional support, the decision to
make my own experiences the topic of the two case studies was an easy one. It
neatly avoids questions of consent and provides me with unrestricted access to
supporting materials. In particular, I had made a habbit of saving interesting
webpages years ago and hence had saved key moments of my interactions with
OpenAI's and Twitter's enforcement processes as screenshots. The Internet
Archive helped fill in missing materials.

By comparison, any survey would have been harder to pull off ethically (e.g., no
review by IRB) and practically (e.g., no students to survey). Alas, the
auto-ethnographic character of the case studies does raise concerns about their
validity. The qualitative nature of both only strengthens the concerns. At the
same time, both case studies are about two corporations' policies and automated
processes, none of which have been created for my benefit and hence clearly go
beyond subjective experience. In other words, for the purposes of this paper, my
interest in these two algorithmic interventions mostly helped to more fully
scope OpenAI's and Twitter's content moderation.

My probing of \DALLE's censor is comparable to studies that probe the security
of internet-facing systems, with one important difference: Whereas probing a
system's security invariably ends up exercising a path infrequently travelled
and hence has a non-zero risk of causing disruption, my probing of \DALLE's
censor was well within the intended use of the system and hence did not pose any
additional risk for disruption. Furthermore, unlike for other text-to-image
systems, generated images are private by default, i.e., only visible to myself
and OpenAI. That allows for considered curation before sharing potentially
upsetting images --- which is just what I did.

Determining the least bad way for handling the unexpected, grotesquely racist
images took a while. I pretty much immediately knew that I didn't want to
include them with the paper. The potential for adding to existing emotional
trauma just seems too big. But that left unaddressed what to do if somebody
wanted to validate my claims or needed the prompts and/or images for their own
research. I found the prospect of me acting as gatekeeper and exercising power
in accepting/rejecting requests unpalatable. In part, that is because I know how
uncomfortable it feels writing such a request as an independent researcher,
without the cachet of an institution to back me up.

After consultation with a close friend, I settled on including prompts and
images with the paper's supplemental materials at
\url{https://github.com/apparebit/penal-colony}. That lets people who want to
see or use the prompts and/or images do so without being asked to justify their
interest. To have recourse in case somebody abuses that, I do assert copyright
for both prompts and images and grant a non-exclusive license only for
``not-for-profit education and scholarship,'' i.e., only within already existing
fair use exemptions. Prompts and images are copyrightable because they are part
of the larger effort to circumvent \DALLE's censor, which requires far more than
a modicum of creativity.

\hidden{
That hints at a tension that permeats this paper, namely the tension between me
as individual and also outsider versus me as academic researcher working within
an institutional tradition. Being personally subjected to Twitter's punishment
ritual certainly made me hyperaware of its punitive condenscension. Yet I turn
to scholarship for demonstrating that this weird anecdote actually has broader
significance. That same scholarship prefers that we maintain an emotional and
intellectual distance --- but grants us authority in return. Yet this weird
anecdote started because I was acting as a daily active shithead --- mind you,
that's an actual industry metric~\cite{Sherman2021}. Though I doubt that
anecdote or metric will grow my academic cachet. Then again, institutional
authority is brittle and prone to double standards, blindspots, and other
failures --- as the paper demonstrates on the example of OpenAI and \DALLE{}.

Consistent with that qualified skepticism of authority, this paper explicitly
favors the perspectives and experiences of individuals over those of
organizations, even if the individual has violated an organization's rules or
country's laws. That is a deliberate reaction to the dehumanizing impact of
organizational policies and algorithmic enforcement described in this paper. For
those same reasons, I appear as myself in this paper and don't hide behind a
pluralis majestatis (uhm, right, see previous paragraph) or similar linguistic
device. They only project but never provide authority. Furthermore, this isn't
an academic affectation. I conducted the research and wrote this paper on my
own, without institutional support and, in particular, without access to an
institutional library.
}


\subsection{Potential Conflicts of Interest}

I worked at Meta n\'ee Facebook as a software engineer from mid 2018 to mid
2019. During that time, I discovered compelling evidence that Meta is cooking
the books when it comes to ad impressions, i.e., the most fundamental
advertising metric~\anon[{[elided]}]{\cite{grimm2022a}}. I also served as paid
consultant to litigation against Meta during the second half of 2022. At the
same time, my work for both roles was unrelated to this research and I do not
have a financial interest in Meta --- or any of the other companies mentioned in
this paper --- beyond, possibly, an indirect interest through index funds.
