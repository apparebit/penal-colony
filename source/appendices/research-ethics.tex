\section{Research Ethics}
\label{app:research-ethics}

Somewhat unusually, I conducted the research for this paper and wrote the paper
without institutional support. That included not having access to an
institutional library. Thankfully, regular search engines do a passable job at
surfacing academic literature. Furthermore, preprint archives and open access
publications make many publications of the last decade or so readily available.
However, that still leaves a large number of publications behind the paywalls of
academic publishers.

I observed per-article rates from \$15 by professional societies to around \$50
(or more) by for-profit publishers, which strike me as excessive. While
publishers often offer better rates for bulk access, they still are entirely
unreasonable given that publishers paid nothing for the hard work of writing,
reviewing, and editing these articles. Instead, I largely relied on Sci-Hub to
access paywalled research articles. Since my prior academic publications did not
benefit from open access options, I have no objections to others accessing them
through Sci-Hub.

My experiemental work on circumventing \DALLE's censor is comparable to security
researchers actively probing the security of internet-facing systems, with two
important differences. First, whereas probing a system's security invariably
ends up exercising paths infrequently travelled and hence has a non-zero risk of
causing system disruption, my probing of \DALLE's censor was well within the
intended use of the system and hence did not pose any additional risk for
disruption. Second, a successful security exploit negatively impacts
confidentiality, integrity, or availability of a system. That is not the case
for my censorship circumvention strategy, which results in violative images for
a given topic and no more. Furthermore, unlike for other text-to-image systems,
generated images are private by default, i.e., only visible to myself and
OpenAI. I not only leveraged that for carefully curating images for publication
but explicitly made aesthetic considerations an integral part of my attack
strategy, never fully testing how for it might go.

Arguably, that also left me somewhat unprepared for when my probing unexpectedly
resulted in grotesquely racist images. I immediately decided that I didn't want
to include them with the paper. Since they were unexpected, their content does
not help support my claims. At the same time, the potential for adding to
existing emotional trauma is too large. But that leaves unaddressed what to do
if somebody wants to validate my claims or evaluate the prompts and/or images
for their own research. I found the prospect of me acting as gatekeeper
unpalatable. That reaction is largely based on my own experiences interacting
with institutional gatekeepers as an independent researcher when compared to as
a faculty member: Writing a request takes considerably more effort to establish
context and credibility but still has markedly lower chances to be granted, as
some gatekeepers seemingly won't even reply to non-institutional requests. I'd
rather avoid putting other researchers into a similar position.

After consultation with a close friend, I settled on including prompts and
images with the paper's supplemental materials at
\url{https://github.com/apparebit/penal-colony}. That lets people who want to
see or use the prompts and/or images do so without being asked to justify their
interest. To have recourse in case somebody abuses unrestricted access, I do
assert copyright for both prompts and images and grant a non-exclusive license
only for ``not-for-profit education and scholarship,'' i.e., only within already
existing fair use exemptions. I'd claim that prompts and images are
copyrightable because they are part of the larger effort to circumvent \DALLE's
censor, which requires far more than a modicum of creativity.


\subsection{Potential Conflicts of Interest}

I worked at Meta n\'ee Facebook as a software engineer from mid 2018 to mid
2019. During that time, I discovered compelling evidence that Meta is cooking
the books when it comes to ad impressions, i.e., the most fundamental
advertising metric, and hence misleading advertisers, investors, and regulators
alike~\cite{grimm2022a}. I also served as paid consultant to litigation against
Meta during the second half of 2022. At the same time, my work for both roles
was unrelated to this research and I do not have a financial interest in Meta
--- or any of the other companies mentioned in this paper --- beyond, possibly,
an indirect interest through index funds.
