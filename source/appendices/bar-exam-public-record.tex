% !TEX root = ../main.tex

\newpage
\section{The California Bar's Misconduct Investigation}
\label{adx:barexam:record}

The California Bar investigated 3,190 out of 9,301 people who took the October
2020 bar exam for possible misconduct, initiated formal proceedings against 432,
and found ``fewer than 50'' actually guilty. This appendix captures the public
record regarding this investigation, comprising, first, an exchange during the
December 2020 meeting of the Committee of Bar
Examiners~\cite{CommitteeOfBarExaminers2020} and, second, a passage in the April
2021 State Auditor's report~\cite{Howle2021}. The Bar did answer journalists'
inquiries, but apparently did not add anything substantial that way.

I asked a friend, who is a member of the California Bar, whether there were any
communications to members about the investigation, but neither they nor two of
their colleagues (who are also members) could find/remember any. In fact, two of
them had not heard about this investigation before.


\subsection{The December 2020 Meeting of the Committee of Bar Examiners}
\label{adx:barexam:meeting}

The relevant exchange during the December 4, 2020 meeting of the Committee of
Bar Examiners for the State Bar of California lasted less than two minutes, from
36:45 to 38:28, out of a four hour meeting. It is between Amy Nunez, the Bar's
Director of Admissions, and Tammy Campbell, a program manager at the Bar. Nunez
graduated from the University of Michigan Law School, has no disciplinary record
in California, but is licensed as in-house counsel only. In other words, she
probably hasn't taken (or passed) the bar exam in California. Campbell has an
MBA from the University of La Verne. I created the transcript by running the
audio through OpenAI's Whisper voice-to-text system~\cite{RadfordKimea2022} and
then fixing several mistakes based on my own review of the conversation.

During the exchange, the current speaker is visible only in a tiny rectangle in
the upper right corner while a slide with this agenda takes up most of the
screen:

\begin{quote}
\textbf{October 2020 California Bar Examination}
\begin{itemize}
\item Who took the exam
\item Video File Review Status
\item Grading Status
\item Post Exam Survey Results
\end{itemize}
\end{quote}

\noindent The transcript follows:

\begin{description}
\item[Nunez] All right, so we're on now with the October bar exam, and I think
    we are going to start with Tammy Campbell.
\item[Campbell] Good morning, everyone. So just to give you guys some numbers on
    how everything went with the October 2020 bar exam, we had 13,082 applicants
    apply for the exam. We had 9,301 that actually sat and took the entire exam.
    Out of that 9,301, we had 8,920 that took that exam online and 3,080 oh
    sorry 381 that actually sat for it in person. So we had a very large number
    that sat online and everything went very well, so that's a good thing to
    see.

    The video review is currently underway for all of the online applicants. We
    are currently reviewing 3,190 applicants that were flagged for their videos.
    That's across various sessions. It doesn't mean every session was flagged,
    but it is 3,190 applicants that we are currently reviewing and everything is
    going well and our expectation is we will be done reviewing the videos by
    December 18th. So a couple more weeks and we'll be able to say we finished
    the review and we'll be able to move on to the next phase of what we need to
    do. So that's all I have for the numbers on the bar exam.
\item[Nunez] Okay, and then the video file review, Tammy?
\item[Campbell] Oh, I just gave all of that information.
\item[Nunez] Oh, sorry about that.
\item[Campbell] Sorry, I kind of segued right into my next part.
\item[Nunez] Oh, no problem. All right, and then with an update [...]
\end{description}

\sectionbreak

\noindent{}The Bar posted the video of the meeting on YouTube on December 11,
2020~\cite{CommitteeOfBarExaminers2020}, ABA Journal reported on the exchange
seven days later, on December 18~\cite{FrancisWardMoran2020}, and Bloomberg Law
picked up the story another four days later, on December 22~\cite{Skolnik2020a}.
Interestingly, ABA Journal didn't link to the official video on YouTube but
another copy posted to Vimeo on December 5 by somebody using the nom de guerre
Fluoxetino Lilly~\cite{Lilly2020}.

By the time ABA Journal's article was published, the Bar had started sending out
so-called ``Chapter 6 notices,'' which asked exam takers to explain exam
irregularities within 10 days. Possible consequences, if the Bar ends up
affirming the notice, range from failing the bar exam to being blocked from
becoming a member, i.e., ending a lawyer's career before it really
began~\cite{TheStateBarOfCalifornia2019}. On December 30, 2020, Bloomberg Law
reported that the Bar had notified 432 exam takers but had also cleared many of
them again~\cite{Skolnik2020}. The Bar released exam results on January 10,
2021~\cite{TheStateBarOfCalifornia2021h}. Two days later, ABA Journal confirmed
that the Bar had sent out a total of 432 Chapter~6 notices. It added that 47 of
them had been affirmed and 6 were still pending~\cite{FrancisWard2021b}.


\subsection{The April 2021 State Auditor's Report}
\label{adx:barexam:audit}

The only other public record regarding the misconduct investigation appears to
be the April 29, 2021 report by the State Auditor~\cite{Howle2021}. Its main
findings, with attention-grabbing capitalization intact, were these:

\begin{enumerate}
\item The State Bar's Changes to Its Discipline System Have Significantly
    Reduced That System's Efficiency.
\item The State Bar's Discipline Report Does Not Provide All Required
    Information, and Its Publishing Date Reduces Its Value to Stakeholders.
\item The State Bar Appropriately Administered the Bar Exam During the COVID-19
    Pandemic, but Its Procurement of Exam Software Did Not Comply With Its
    Policy.
\end{enumerate}

The discussion of the third finding includes the following three paragraphs
regarding the October 2020 bar exam. Footnotes also are in the original.

\sectionbreak

\noindent{}In addition, following the Supreme Court's August 2020 order
modifying how and when it should administer the bar exam, the State Bar amended
an existing contract with its software vendor, ExamSoft, to obtain remote
proctoring services for the October 2020 exam. The State Bar had previously
signed a five-year, \$3 million contract with ExamSoft in May 2020 providing it
with software that applicants install on their personal computers or use on the
State Bar's computers in order to take the exam.\footnote{The records that the
State Bar provided indicate that it has contracted with ExamSoft since at least
2003 to provide software for bar exams.} This contract provided the software for
10 bar exam dates expected to occur from July 2020 through February
2025.\footnote{The State Bar pays ExamSoft according to the total number of
individuals who register to take the bar exam; therefore, the actual amount the
State Bar pays may be more or less than the original contract amount.} The State
Bar amended this contract in August 2020 to include ExamSoft's verifying
applicants' identity, recording applicants for the duration of the exam, and
reviewing the recordings to identify suspicious behavior. This amendment was
exclusive to the October 2020 bar exam and cost the State Bar an additional
\$830,000.

According to the State Bar, about 8,900 applicants of the 9,300 applicants who
took the exam in October 2020 did so remotely.\footnote{The State Bar provided
in‑person examinations on a case‑by‑case basis to applicants requesting certain
testing accommodations and to applicants who indicated they lacked a testing
environment conducive to taking the exam.} Subsequently, the State Bar reviewed
nearly 3,200 videos that the software and human review had flagged for possible
violations, such as the applicants' leaving the camera's view or using other
electronic devices, and it ultimately found fewer than 50 violations of
examination rules and policies. The State Bar signed another contract amendment
with ExamSoft in January 2021 for an additional \$1.3 million to obtain remote
proctoring services for the February 2021 exam after the Supreme Court issued an
order in November 2020 directing it to also administer that bar exam remotely.

The State Bar's actions effectively implemented the Supreme Court's orders
related to the temporary licensure program and the remote administration of the
October 2020 bar exam. Its actions provided eligible graduates an opportunity to
practice law in California under the supervision of an eligible California
attorney while waiting to take the bar exam. Further, the State Bar administered
the bar exam remotely for the first time while taking steps to preserve the bar
exam's integrity through the acquisition of additional services to verify
applicants' identity and to monitor for suspicious behavior. However, as we
describe in the next section, the State Bar should have documented that it
received the best value when contracting for these services.
