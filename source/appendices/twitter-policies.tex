% !TEX root = ../main.tex

\newpage
\section{Twitter's Policy on Abusive Behavior}
\label{adx:twitter:abusive-behavior}

The text of Twitter's policy on abusive behavior as of 5 September 2022 follows.
It preserves the structure and formatting of the original, with links pointing
to the Internet Archive. The current version of the policy is available at
\url{https://help.twitter.com/en/rules-and-policies/abusive-behavior}.

\sectionbreak

\noindent\href{https://web.archive.org/web/20220905021323/https://help.twitter.com/en/rules-and-policies/twitter-rules.html}{Twitter
Rules}: You may not engage in the targeted harassment of someone, or incite
other people to do so. We consider abusive behavior an attempt to harass,
intimidate, or silence someone else's voice.


\subsection{Rationale}

On Twitter, you should feel safe expressing your unique point of view. We
believe in freedom of expression and open dialogue, but that means little as an
underlying philosophy if voices are silenced because people are afraid to speak
up.

In order to facilitate healthy dialogue on the platform, and empower individuals
to express diverse opinions and beliefs, we prohibit behavior that harasses or
intimidates, or is otherwise intended to shame or degrade others. In addition to
posing risks to people's safety, abusive behavior may also lead to physical and
emotional hardship for those affected.

Learn more about our approach to
\href{https://web.archive.org/web/20220905021323/https://help.twitter.com/en/rules-and-policies/enforcement-philosophy.html}{policy
development and our enforcement philosophy}.


\subsection{When This Applies}

Some Tweets may seem to be abusive when viewed in isolation, but may not be when
viewed in the context of a larger conversation. When we review this type of
content, it may not be clear whether it is intended to harass an individual, or
if it is part of a consensual conversation. To help our teams understand the
context of a conversation, we may need to hear directly from the person being
targeted, to ensure that we have the information needed prior to taking any
enforcement action.

We will review and take action against reports of accounts targeting an
individual or group of people with any of the following behavior within Tweets
or Direct Messages. For accounts engaging in abusive behavior on their profile,
please refer to our
\href{https://web.archive.org/web/20220905021323/https://help.twitter.com/en/rules-and-policies/abusive-profile.html}{abusive
profile policy}. For behavior targeting people based on their race, ethnicity,
national origin, sexual orientation, gender, gender identity, religious
affiliation, age, disability, or serious disease, this may be in violation of
our
\href{https://web.archive.org/web/20220905021323/https://help.twitter.com/en/rules-and-policies/hateful-conduct-policy.html}{hateful
conduct policy}.


\vspace{.5em}\noindent\textbf{Violent Threats}

\noindent We prohibit content that makes violent threats against an identifiable
target. Violent threats are declarative statements of intent to inflict injuries
that would result in serious and lasting bodily harm, where an individual could
die or be significantly injured, e.g., ``I will kill you.''

\textbf{Note}: We have a zero tolerance policy against violent threats. Those
deemed to be sharing violent threats will face immediate and permanent
suspension of their account.


\vspace{.5em}\noindent\textbf{Wishing, hoping, or calling for serious harm on a
    person or group of people}

\noindent We do not tolerate content that wishes, hopes, promotes, incites, or
expresses a desire for death, serious bodily harm or serious disease against an
individual or group of people. This includes, but is not limited to:

\begin{itemize}
\item Hoping that someone dies as a result of a serious disease e.g., ``I
    hope you get cancer and die.''
\item Wishing for someone to fall victim to a serious accident e.g., ``I
    wish that you would get run over by a car next time you run your
    mouth.''
\item Saying that a group of individuals deserves serious physical injury
    e.g., ``If this group of protesters don't shut up, they deserve to be
    shot.''
\end{itemize}


\vspace{.5em}\noindent\textbf{About wishes of harm exceptions on Twitter}

\noindent We recognize that conversations regarding certain individuals credibly
accused of severe violence may prompt outrage and associated wishes of harm. In
these limited cases, we will request the user to delete the Tweet without any
risk of account penalty, strike, or suspension. Examples are, but not limited
to:

\begin{itemize}
\item ``I wish all rapists to die.''
\item ``Child abusers should be hanged.''
\end{itemize}


\vspace{.5em}\noindent\textbf{Unwanted sexual advances}

\noindent While some
\href{https://web.archive.org/web/20220905021323/https://help.twitter.com/en/rules-and-policies/media-policy.html}{consensual
nudity and adult content is permitted} on Twitter, we prohibit unwanted sexual
advances and content that sexually objectifies an individual without their
consent. This includes, but is not limited to:

\begin{itemize}
\item sending someone unsolicited and/or unwanted adult media, including images, videos, and GIFs;
\item unwanted sexual discussion of someone’s body;
\item solicitation of sexual acts; and
\item any other content that otherwise sexualizes an individual without
    their consent.
\end{itemize}


\vspace{.5em}\noindent\textbf{Using insults, profanity, or slurs with the purpose
    of harassing or intimidating others}

\noindent We take action against the use of insults, profanity, or slurs to
target others. In some cases, such as (but not limited to) severe, repetitive
usage of insults or slurs where the primary intent is to harass or intimidate
others, we may require Tweet removal. In other cases, such as (but not limited
to) moderate, isolated usage of insults and profanity where the primary intent
is to harass or intimidate others, we may limit Tweet visibility as further
described below. Please also note that while some individuals may find certain
terms to be offensive, we will not take action against every instance where
insulting terms are used.


\vspace{.5em}\noindent\textbf{Encouraging or calling for others to harass an
    individual or group of people}

\noindent We prohibit behavior that encourages others to harass or target
specific individuals or groups with abusive behavior. This includes, but is not
limited to; calls to target people with abuse or harassment online and behavior
that urges offline action such as physical harassment.


\vspace{.5em}\noindent\textbf{Denying mass casualty events took place}

\noindent We prohibit content that denies that mass murder or other mass
casualty events took place, where we can verify that the event occured [sic],
and when the content is shared with abusive intent. This may include references
to such an event as a ``hoax'' or claims that victims or survivors are fake or
``actors.'' It includes, but is not limited to, events like the Holocaust,
school shootings, terrorist attacks, and natural disasters.


\vspace{.5em}\noindent\textbf{Do I need to be the target of this content for it
    to be reviewed for violating the Twitter Rules?}

\noindent No, we review both first-person and bystander reports of such content.


\vspace{.5em}\noindent\textbf{Consequences}

\noindent When determining the penalty for violating this policy, we consider a
number of factors including, but not limited to, the severity of the violation
and an individual's previous record of rule violations. The following is a list
of potential enforcement options for content that violates this policy:

\begin{itemize}
\item Downranking Tweets in replies, except when the user follows the Tweet author.
\item Making Tweets ineligible for amplification in Top search results and/or on timelines for users who don't follow the Tweet author.
\item Excluding Tweets and/or accounts in email or in-product recommendations.
\item Requiring Tweet removal.
    \begin{itemize}
    \item For example, we may ask someone to remove the violating content
        and serve a period of time in read-only mode before they can Tweet
        again. Subsequent violations will lead to longer read- only periods
        and may eventually result in permanent suspension.
    \end{itemize}
\item Suspending accounts whose primary use we've determined is to engage in abusive behavior as defined in this policy, or who have shared violent threats.
\end{itemize}

Learn more about
\href{https://web.archive.org/web/20220905021323/https://help.twitter.com/en/rules-and-policies/enforcement-options.html}{our
range of enforcement options}.

If someone believes their account was suspended in error, they can
\href{https://web.archive.org/web/20220905021323/https://help.twitter.com/forms/general?subtopic=suspended}{submit
an appeal}.
